\documentclass[hidelinks, 12pt,letterpaper]{article}
%%%%%%%%%%%%%%%%%%%%%%%%%%%%%%%%%%%%%%%%%%%%%%%%%%%%%%%%%%%%%%%%%%%%%%%%%%%%%%%%%%%%%%%%%%%%%%%%%%%%%%%%%%%%%%%%%%%%%%%%%%%%
\usepackage{amsmath}
%\usepackage{graphicx,fancyheadings}
\usepackage{fancyhdr}
\pagestyle{fancy}
\usepackage{graphics}
\usepackage{amssymb}
\usepackage{verbatim}
\usepackage{setspace}
\usepackage{ulem} 

\usepackage{amssymb}
\usepackage{graphicx}
\usepackage{amsmath}
\usepackage{verbatim}
\usepackage{setspace}
\usepackage{ulem}
\usepackage{textpos}
\usepackage{changepage}
\usepackage{url}
\usepackage{hyperref}

\usepackage{color}
\usepackage{xcolor}


%\renewcommand{\sout}[1]{}
%\renewcommand{\bf}{}

\usepackage[round]{natbib}
% \usepackage[backend=biber,style=apa,autocite=inline]{biblatex} \DeclareLanguageMapping{english}{english-apa}
% \addbibresource{./sshrc2023shipping.bib}


%\lhead{} \chead{} \cfoot{}
\renewcommand{\headrulewidth}{0pt}
\renewcommand{\footrulewidth}{0pt}
\renewcommand{\baselinestretch}{1.1}
%\textheight 8in \textwidth 6in \oddsidemargin 0.15in
%\evensidemargin 0in \headheight 14.5pt \headwidth 6in \parskip 4pt

\pagestyle {empty}

\setlength{\topmargin}{0.0cm} \setlength{\headheight}{0cm}
\setlength{\headsep}{0.3cm} \setlength{\topskip}{0cm}
\setlength{\textheight}{22.5cm} \setlength{\textwidth}{16.6cm}
\setlength{\oddsidemargin}{0cm} \setlength{\evensidemargin}{0cm}
\setlength{\footskip}{0.5cm}
\newcommand{\indep}{\perp \!\!\! \perp}


\newcommand{\Red}{\color{red}}
\newcommand{\Blue}{\color{blue}}
\newcommand{\ve}{\varepsilon}

\def\bs{\boldsymbol}

%\setlength{\topmargin}{-0.2cm} \setlength{\headheight}{0cm}
%\setlength{\headsep}{0.3cm} \setlength{\topskip}{0cm}
%\setlength{\textheight}{22cm} \setlength{\textwidth}{16.2cm}
%\setlength{\oddsidemargin}{0cm} \setlength{\evensidemargin}{0cm}
%\setlength{\footskip}{1cm}

\usepackage[letterpaper]{geometry}


\graphicspath{{../images/shared}}

\rhead{Hiroyuki Kasahara \hspace{-1.7cm} }
%\lfoot{}
\cfoot{\thepage \hspace{-2cm}}



%\setcounter{page}{10}

\begin{document}
\bibliographystyle{asa}

% Challenge—The aim and importance of the endeavour (40\%):
% \begin{itemize}
%   \item originality, significance and expected contribution to knowledge;
%   \item appropriateness of the literature review;
%   \item appropriateness of the theoretical approach or framework;
%   \item appropriateness of the methods/approach;
%   \item quality of training and mentoring to be provided to students, emerging scholars and other highly qualified personnel, and opportunities for them to contribute; and
%   \item potential for the project results to have influence and impact within and/or beyond the social sciences and humanities research community.
% \end{itemize}
    
% Feasibility—The plan to achieve excellence (20\%):
% \begin{itemize}
%   \item appropriateness of the proposed timeline, and probability that the objectives will be met;
% \end{itemize}

\begin{center}
\textbf{Detailed Description of Proposed Research: }\vspace{-0.25cm}
\end{center}

\noindent \textbf{Project: Quantifying the response of maritime shipping CO$_2$ emissions to demand shocks} 
\smallskip 

\noindent \textbf{Objective:} To estimate the elasticity of CO$_2$ emissions from maritime shipping with respect to international trade using the COVID pandemic demand shock as a source of significant variation.
\smallskip 

\noindent \textbf{Context:} 
\textbf{Overview}
Global trade is intricately linked with maritime shipping, which carries over 80\% of the volume of all traded goods and around 70\% of their value \citep{unctad2017review}. 
% The importance of the shipping industry has become particularly apparent in recent years as disruptions ranging from the blockage of the Suez Canal to widespread COVID-related port slowdowns have snarled supply chains world-wide.
At the same time, maritime ships contribute roughly 3\% of global CO$_2$ emissions, roughly equal to the total emissions of Germany \citep*{faber2020fourth}. These emissions lie outside the scope of national emissions tallies, and fall instead under the jurisdiction the International Maritime Organization (IMO), which has set a target of a 50\% reduction by 2050.
\textcolor{blue}{mention difficulty of decarbonization?}
The stringency of abatement actions required to meet this goal clearly depends on the how trade will evolve over the coming decades. A continuation of the trend of increasing trade would make this goal much more difficult to hit, while a reversion to more protectionist policies would ease the challenge. Faced with this uncertainty, the IMO and its consituent countries are nevertheless developing and implementing policies to reduce shipping emissions, with new efficiency regulations being phased in this year. We aim to provide new quantitative evidence of the relationship between international trade and maritime shipping emissions in order to better inform policy makers. To do so, we will exploit the large variation in demand for shipping that resulted from the COVID pandemic.

The three largest sectors of maritime shipping, jointly accounting for over half of maritime emissions, are containerships, bulk carriers, and tankers. These ships transport, respectively,  containerized goods (often manufactured goods), dry bulk goods (e.g. coal, ores, grains), and bulk liquids (primarily oil). As such, they contribute to diverse links of the overall supply chain and are impacted differently by fluctuations in trade. Furthermore, each sector has distinct market characteristics. For example, containerships typically operate with fixed routes and schedules, while bulk carriers tend to operate much more flexibly. Ownership structures reflect these differences as well, with the containership sector being highly concentrated and the bulk carrier sector highly \textit{un}concentrated.

Despite their differences, the various sectors share some common mechanisms by which they adjust to demand. Because new ships cost tens to hundreds of millions of dollars, last for over 25 years, and take two to five years to build, adjustments in fleet capacity are slow to occur. On shorter time horizons, shipping supply adjusts to changing levels of demand primarily through a combination of changing travel speed and temporarily idling ships, though quantifying these adjustments is an active area of research (see \citet*{adland2018dynamic, ollila2022effect,assmann2015missing}). CO$_2$ emissions are directly related to fuel consumption, which depends roughly cubicly on speed, meaning that the short run elasticity of emissions to demand may be quite large. 

Quantifying this elasticity is challenging for a number of reasons. First, a ship's fuel consumption depends on various factors, including its size and age (newer and larger ships tend to be more efficient) and the existing fleet is extremely heterogeneous. As an illustration, \autoref{fig:distribution} shows the existing fleet distribution for bulk carriers below 100,000 DWT in size (this excludes the largest classes up to just over 400,000 DWT). 
\textcolor{blue}{Is this a good figure to include? Is it better to use MRV emissions plots instead?} 
As such, ships adapt differently and which ships change speed or idle is important for determining overall emissions.
Shifts in the geographic distribution of trade will further impact emissions through mechanisms such as changes in shipping distances, backhaul effects (ships travel empty on certain routes due to sector-specific trade imbalances), and economies of scale (route-specific trade volumes and port infrastructure determine the size of ships used).


\begin{figure}[h]
  \centering
  \includegraphics[width = 0.75\textwidth]{WFR_Bulkers_Exploration_Size_Built_heatmap.png}
  \caption{Number of new ships by size (Dwt) and built year \textcolor{blue}{fix labels}
}
  \label{fig:distribution}
\end{figure}

The IMO has introduced three separate emissions regulations. The first, implemented in 2013, is a minimum efficiency requirement for new ships, with stringency increasing over time and future levels yet to be determined. As of the beginning of 2023, a similar regulation now applies to all existing ships and an additional regulation has be added that requires operational efficiency of each ship to be continuously improved moving forward. Uncertainty surrounding these regulations, as well as future technological developments has led to reduced new ship building activity in the past years. With a diminished extrinsic margin for adjustment in addition to the already slow natural fleet development, this places further importance on quantifying the intrinsic margin for predicting the path of shipping emissions. 

The most extensive existing literature regarding shipping emissions comes from the IMO itself, in cooperation with a handful of related industry organizations. In particular, the Fourth IMO GHG Study 2020 \citep{faber2020fourth} details both bottom-up and top-down methodologies for calculating emissions. Their bottom-up approach relies on high frequency tracking data and has been developed and employed by various authors (e.g. \citet{olmer2017greenhouse, johansson2017global, jalkanen2009modelling, van2018spatially}). All ships are equipped with automatic identification system (AIS) transceivers which transmit information about the location and speed of each ship every few minutes. In order to estimate emissions, this information is combined with ship fuel consumption ratings and aggregated.

With regards to relating trade to shipping activity, \citet{brancaccio2018impact} explore the elasticity of trade with respect to ship fuel costs. Our work will be some of the first to seriously explore the relationship in the opposite direction - from trade to emissions. To the best of our knowledge, our work will be the first to utilize actual reported emissions to empirically estimate ship efficiencies on a large scale, which allows for more of the previously mentioned channels to be captured. In addition, our approach is novel in its use of machine learning to extrapolate efficiencies for ships without reported emissions. Furthermore, we are not aware of any literature yet that exploits the large variation in shipping activity due to COVID to explore emissions. With these contributions, we hope to provide important quantitative estimates to help inform policy makers in assessing the effectiveness of emissions regulations and setting their stringency levels going forward.


\smallskip 

\noindent \textbf{Methodology:} We proceed in two stages. We first develop high-frequency disaggregated emissions estimates and then we link these to trade volumes and patterns. The first stage relies on three key datasets that we have obtained, AIS tracking data, a fleet register, and individual emissions reports.

We have obtained hourly AIS tracking data for the entire fleets of bulk carriers and containerships from the beginning of 2019 to the end of 2021. This includes information on speed, location, and draft, which can be used to determine whether a ship is carrying cargo or not. This data is then matched to the World Fleet Register from Clarksons Research, which is a virtually complete listing of all large merchant ships. It includes basic  information on each ship, including built year, size, and type, and for many ships includes highly detailed technical characteristics such as hull dimensions, engine power, propeller details, etcetera. Finally, this can be further linked to publicly available data collected through the European Union's Monitoring, Reporting, and Verification (MRV) regulation, which provides annual fuel consumption and emissions for trips into and out of the EU. This data begins in 2018 and naturally includes only ships with portcalls in the European Union in a given year.

Our methodology builds on that of the IMO as detailed in \citet{faber2020fourth} and follows closely the data cleaning and matching procedures described therein. However, whereas they use nominal fuel consumption values corresponding to rather coarse ship size- and age-bins we propose to empirically estimate more ship-specific fuel efficiencies. In doing so, we hope to better reflect variation in fuel consumption under actual operating conditions. Our procedure consists of four steps: First, we estimate fuel efficiency of the subset of ships reporting in the MRV dataset. Then, we extrapolate these efficiencies to non-reporting ships based on ship characteristics. Given ship efficiencies, we can calculate a high frequency emissions estimate for each ship. Finally, these estimates can be aggregated at any desired level. To our knowledge, this will be the first work to employ the MRV data to estimate fuel efficiency. \citet{uge2020estimation} also link MRV data with AIS data, but they use it in the opposite sense, namely to validate reported emissions in the MRV.

Because the MRV data encompasses only trips in and out of the EU, in order to estimate fuel efficiency, we must first detect voyages from the tracking data and identify those that involve a portcall in the EU. We detect stops based on a ship speed threshold and a location near to land and denote a voyage as a trip between any two stops. In order to ensure the accuracy of our data, we then use data only for ship-year observations for which the total distance of detected trips to/from the EU agrees closely with the distance reported in the MRV data. With this travel history constructed, we calculate a proxy for the travel work performed by a ship in a given year as the sum of its speed squared multiplied by the distance travelled between every pair of observations in the AIS data. The fuel efficiency is then calculated as the reported annual fuel consumption divided by the inferred annual travel work. 

A limitation of this approach is that the fuel consumption data is annual and there may be significant error in calculating the travel work over such a long time period. Furthermore, as described above, this does not incorporate the effects of laden status or weather. We can augment the travel work calculation to include the draft level and possibly wind and wave speeds using detailed weather data like that used by \citet{brancaccio2020geography}, however this requires more assumptions regarding hull shape. The advantage of this approach over that used by \citet{faber2020fourth} is that it relies less on theoretical assumptions. As noted by \citet{olmer2017greenhouse}, details such as the engine power curve (how fuel consumption varies away from design speed), hull-roughening, and hull-fouling are hard to predict or attribute. Our estimates would incorporate these effects at the average level of observed ships.

To date, we have estimated efficiencies using this procedure for bulk carriers. We have further constructed a simple linear predictive model for efficiency extrapolation, regressing fuel efficiency on a set of ship characteristics (using logs of all variables) as well as built-year fixed effects. 
% \begin{equation}
% \log\left(
%     \frac{fuel~consumption}{Dwt \cdot \sum_{x \in X}  \cdot s_x^2 \cdot x}
% \right)_{it}    
%         = \delta age_{it} + \boldsymbol{\beta}log(Z_i) + \varepsilon_{it}
% \end{equation}
The coefficients for the built-year fixed effects are plotted in \autoref{fig:efficiency} and indicate that efficiency after controlling for size is surprisingly flat for ships built before roughly 2013, after which efficiency improved. This agrees qualitatively with the analysis of evolution of new ship efficiency from \citet{faber2015historical} (see Figure 15).

\begin{figure}[h]
  \centering
  \includegraphics[width = 0.75\textwidth]{Efficiency_Regression_Size_Age_Coeffs_3.pdf}
  \caption{\textcolor{blue}{update, tidy, fix labels}}
  \label{fig:efficiency}
\end{figure}

Our next step is to assess the quality of extrapolation in a more systematic manner using cross-validation on randomly selected training and testing subsets. We are also developing an alternative, more flexible neural network model for efficiency extrapolition, and will compare its performance to the simple linear model. Finally, this exercise will be repeated for containerships.

% Trade
%% Data
COVID variation
\textbf{Trade data} Bilateral trade data from...

\textbf{Method of linking to emissions to trade}
Begin with global, then incorporate geographical variation


\pagebreak
Potential data purchases:
\begin{itemize}
  \item expand time series of AIS tracking data beyond 2021
  \item AIS tracking data for tankers
  \item bilateral trade
\end{itemize}

\pagebreak
  
% \printbibliography[heading = subbibliography]
\singlespace{
\bibliography{sshrc2023shipping}
}

 

\end{document}