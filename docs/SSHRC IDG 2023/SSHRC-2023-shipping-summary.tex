\documentclass[hidelinks, 12pt,letterpaper]{article}
%%%%%%%%%%%%%%%%%%%%%%%%%%%%%%%%%%%%%%%%%%%%%%%%%%%%%%%%%%%%%%%%%%%%%%%%%%%%%%%%%%%%%%%%%%%%%%%%%%%%%%%%%%%%%%%%%%%%%%%%%%%%
\usepackage{amsmath}
%\usepackage{graphicx,fancyheadings}
\usepackage{fancyhdr}
\pagestyle{fancy}
\usepackage{graphics}
\usepackage{amssymb}
\usepackage{verbatim}
\usepackage{setspace}
\usepackage{ulem}

\usepackage{amssymb}
\usepackage{graphicx}
\usepackage{amsmath}
\usepackage{verbatim}
\usepackage{setspace}
\usepackage{ulem}
\usepackage{textpos}
\usepackage{changepage}
\usepackage{url}
\usepackage{hyperref}

\usepackage{color}
\usepackage{xcolor}
 

%\renewcommand{\sout}[1]{}
%\renewcommand{\bf}{}

\usepackage[round]{natbib}
% \usepackage[backend=biber,style=apa,autocite=inline]{biblatex} \DeclareLanguageMapping{english}{english-apa}
% \addbibresource{./sshrc2023shipping.bib}


%\lhead{} \chead{} \cfoot{}
\renewcommand{\headrulewidth}{0pt}
\renewcommand{\footrulewidth}{0pt}
\renewcommand{\baselinestretch}{1.1}
%\textheight 8in \textwidth 6in \oddsidemargin 0.15in
%\evensidemargin 0in \headheight 14.5pt \headwidth 6in \parskip 4pt

\pagestyle {empty}

\setlength{\topmargin}{0.0cm} \setlength{\headheight}{0cm}
\setlength{\headsep}{0.3cm} \setlength{\topskip}{0cm}
\setlength{\textheight}{22.5cm} \setlength{\textwidth}{16.6cm}
\setlength{\oddsidemargin}{0cm} \setlength{\evensidemargin}{0cm}
\setlength{\footskip}{0.5cm}
\newcommand{\indep}{\perp \!\!\! \perp}


\newcommand{\Red}{\color{red}}
\newcommand{\Blue}{\color{blue}}
\newcommand{\ve}{\varepsilon}

\def\bs{\boldsymbol}

%\setlength{\topmargin}{-0.2cm} \setlength{\headheight}{0cm}
%\setlength{\headsep}{0.3cm} \setlength{\topskip}{0cm}
%\setlength{\textheight}{22cm} \setlength{\textwidth}{16.2cm}
%\setlength{\oddsidemargin}{0cm} \setlength{\evensidemargin}{0cm}
%\setlength{\footskip}{1cm}

\usepackage[letterpaper]{geometry}

\usepackage{graphicx}
\graphicspath{{../images/shared/}}


 \rhead{Hiroyuki Kasahara \hspace{-1.7cm} }
 %\lfoot{}
 \cfoot{\thepage \hspace{-2cm}}



%\setcounter{page}{10}

\begin{document}
\bibliographystyle{asa}
 
CO2 emissions are a primary driver of global warming and present one of humankind's most pressing challenges. Maritime shipping contributes a third of CO2 emissions from all trade-related activities, which corresponds to roughly 3\% of global CO2 emissions. The International Maritime Organization (IMO) has set a target of a 50\% reduction of CO2 emissions from maritime shipping by 2050. The stringency of abatement actions required to meet this goal depends on how trade will evolve over the coming decades, and a thorough understanding of how an increase in trade affects CO2 emissions from maritime shipping is essential for effective policy. 

In this project, we measure the change in shipping CO2 emissions over years using high-frequency satellite data of ships’ movements (hourly AIS tracking data for the entire fleets in the world). By exploiting the large variation in shipping during the COVID pandemic, we estimate the short- to medium-run elasticity of CO2 emissions from maritime shipping with respect to international trade. Using the estimated elasticities, we assess the impact of policy regulations on the worldwide CO2 emissions. 

We combine three datasets. First, hourly AIS tracking data includes information on speed, location, and draft (the vertical distance between the waterline and the bottom of the hull), which can be used to determine whether a ship is carrying cargo or not. We match this data to the World Fleet Register, which includes built year, size, type, and other technical characteristics such as hull dimensions, engine power, propeller details, etc.. Finally, we link this to data from the EU’s MRV regulation, which provides annual fuel consumption and emissions for trips into and out of the EU. 

From the AIS data, we identify all trips between a pair of ports for each ship. We estimate how fuel efficiency is determined by ship characteristics (age, size, etc.) and operating conditions (speed, draft, weather) using the fuel consumption data for EU trips from the MRV dataset. We then extrapolate these efficiencies to non-reporting ships—ships that did not stop at an EU port—based on their ship characteristics and operating conditions. Once fuel efficiencies for all trips across all ships are estimated, we may estimate fuel consumptions and their associated CO2 emissions for all trips of any ship. 

We compute the worldwide CO2 emissions within each month over years by aggregating fuel consumptions across all trips. Fuel consumption for each port pair is estimated by aggregating all trips taken from the origin to the destination port. By aggregating them at the origin-destination country pair, we decompose a change in the worldwide CO2 emission as the sum of a change in directional bilateral trade flows across different countries and directions. The directionality is important because many carriers travel without cargo due to trade imbalances, and we account for it by identifying ship loading and unloading at each port using the draft from AIS data while using monthly product-level bilateral trade data from UN Comtrade, Eurostat, and the US census as supplementary information. 

We estimate the elasticity of CO2 emissions specific to each ship category and origin-destination pair and then compute the elasticity of CO2 emissions with respect to trade volume from each origin country to each destination country by aggregating the elasticities across different ship categories using their observed empirical shipping weights, where route-specific draft is used to adjust for the capacity utilization to account for trade imbalances. Using the estimated elasticities, we evaluate the impact of two policy regulations on CO2 emissions: the effect of regulating the maximum speed and the effect of regulating unloaded trips.






 
 

 


\end{document}