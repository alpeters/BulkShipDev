\documentclass[hidelinks, 12pt,letterpaper]{article}
%%%%%%%%%%%%%%%%%%%%%%%%%%%%%%%%%%%%%%%%%%%%%%%%%%%%%%%%%%%%%%%%%%%%%%%%%%%%%%%%%%%%%%%%%%%%%%%%%%%%%%%%%%%%%%%%%%%%%%%%%%%%
\usepackage{amsmath}
%\usepackage{graphicx,fancyheadings}
\usepackage{fancyhdr}
\pagestyle{fancy}
\usepackage{graphics}
\usepackage{amssymb}
\usepackage{verbatim}
\usepackage{setspace}
\usepackage{ulem}

\usepackage{amssymb}
\usepackage{graphicx}
\usepackage{amsmath}
\usepackage{verbatim}
\usepackage{setspace}
\usepackage{ulem}
\usepackage{textpos}
\usepackage{changepage}
\usepackage{url}
\usepackage{hyperref}

\usepackage{color}
\usepackage{xcolor}
 

%\renewcommand{\sout}[1]{}
%\renewcommand{\bf}{}

\usepackage[round]{natbib}
% \usepackage[backend=biber,style=apa,autocite=inline]{biblatex} \DeclareLanguageMapping{english}{english-apa}
% \addbibresource{./sshrc2023shipping.bib}


%\lhead{} \chead{} \cfoot{}
\renewcommand{\headrulewidth}{0pt}
\renewcommand{\footrulewidth}{0pt}
\renewcommand{\baselinestretch}{1.1}
%\textheight 8in \textwidth 6in \oddsidemargin 0.15in
%\evensidemargin 0in \headheight 14.5pt \headwidth 6in \parskip 4pt

\pagestyle {empty}

\setlength{\topmargin}{0.0cm} \setlength{\headheight}{0cm}
\setlength{\headsep}{0.3cm} \setlength{\topskip}{0cm}
\setlength{\textheight}{22.5cm} \setlength{\textwidth}{16.6cm}
\setlength{\oddsidemargin}{0cm} \setlength{\evensidemargin}{0cm}
\setlength{\footskip}{0.5cm}
\newcommand{\indep}{\perp \!\!\! \perp}


\newcommand{\Red}{\color{red}}
\newcommand{\Blue}{\color{blue}}
\newcommand{\ve}{\varepsilon}

\def\bs{\boldsymbol}

%\setlength{\topmargin}{-0.2cm} \setlength{\headheight}{0cm}
%\setlength{\headsep}{0.3cm} \setlength{\topskip}{0cm}
%\setlength{\textheight}{22cm} \setlength{\textwidth}{16.2cm}
%\setlength{\oddsidemargin}{0cm} \setlength{\evensidemargin}{0cm}
%\setlength{\footskip}{1cm}

\usepackage[letterpaper]{geometry}

\usepackage{graphicx}
\graphicspath{{../images/shared/}}


 \rhead{Hiroyuki Kasahara \hspace{-1.7cm} }
 %\lfoot{}
 \cfoot{\thepage \hspace{-2cm}}



%\setcounter{page}{10}

\begin{document}
\bibliographystyle{asa}
 
 
Maritime shipping transports over 80\% of the volume of all traded goods but contribute about 3\% of global CO2 emissions, roughly equal to the total emissions of Germany.  The International Maritime Organization (IMO) has set a target of a 50\% reduction of CO2 emission from maritime shipping by 2050. The stringency of abatement actions required to meet this goal clearly depends on how trade will evolve over the coming decades, and a thorough understanding of how an increase in trade affects CO2 emissions from international shipping is essential for effective policy.  


World merchandise trade decreased by more than 10 percent in the first three months of the pandemic before recovering over the following two years.  The proposed project first aims at measuring the worldwide CO2 emission from maritime shipping before and during the COVID pandemic using high-frequency satellite data of movements of all registered ships in the world. By exploiting a large cross-sectional and time-series variation in trade volumes, we analyze how a change in CO2 emissions from maritime shipping is related to a change in trade volumes between country pairs during the COVID pandemic. We estimate the short- to medium-run elasticity of CO2 emissions from maritime shipping with respect to international trade and how the elasticities depend on the ship speed and capacity utilization. Using the estimated elasticities of CO2 emission with respect to trade volumes, we evaluate the impact of implementing  the following two policy regulations on CO$_2$ emissions. First, we evaluate the effect of regulating the maximum speed of ships on CO$_2$ emissions. Second, we evaluate the effect of regulating unloaded trips in the context of trade imbalance. 
 
 

 


\end{document}