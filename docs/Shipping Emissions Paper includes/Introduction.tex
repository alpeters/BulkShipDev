\section{Introduction}\label{MLintro}

% Motivation:
% - Shipping emissions are a significant contributor to global CO$_2$ emissions.
% - Traditional estimates rely on engineering approximations that may not entirely capture actual fuel use.
% - Higher frequency estimates (typically published with lag)
% - Existing estimates may suffer from poor data quality
The most recent official estimate of \ac{GHG} emissions from maritime shipping places the sector's contribution at around 3\% of global emissions \parencite{faber2020fourth}. Since this estimate was published, various events have significantly impacted the shipping industry, including the COVID-19 pandemic and the implementation of new measures aimed at improving ship efficiency. Such events have almost certainly had immediate effects on emissions, and quantifying their magnitude can provide insights into the effectiveness of actual policy measures and how to improve them. To do so, it is essential to have accurate and timely estimates of emissions. In this paper, we develop a prediction methodology that augments the industry-standard method based on engineering relationships with machine learning in order to provide more accurate estimates. These estimates can be updated and published in real-time, providing invaluable information for policymakers, researchers, and industry stakeholders alike.

% What we do:
% - Provide a machine learning approach to more accurately estimate CO$_2$ emissions
% - Higher frequency estimates (typically published with lag)
% - Take emissions reports as truthful and accurate
% - Predict fuel consumptions directly, providing bottom-up approach as predictor (perhaps a sort of featuring engineering)
% - Dealing with missing data for extrapolation
% - We are able to explain x\% of variation in residual
% - Better untracked (no AIS data) predictions???

To develop and evaluate our model of shipping emissions, we leverage publicly available emissions reporting data from the \ac{MRV} program implemented by the \ac{EU}. \Ac{CO2} emissions are directly proportional to fuel consumption,\footnote{For example the most common fuel, \ac{HFO}, emits 3.114 kg \ac{CO2} per kg of fuel.} and we therefore take (log) reported annual ship fuel consumption as our target variable. We evaluate various machine learning algorithms, using data on both ship characteristics and ship activity as predictor variables. Crucially, we include features derived from the engineering-based fuel consumption calculations used by the \ac{IMO}. These features use hourly ship activity observations from \ac{AIS} data, including location, speed, and draft,\footnote{Draft is the vertical distance between the waterline and the bottom of the hull. It is a function of how heavily laden the ship is and affects fuel consumption.} to construct annual aggregate features in a manner that is consistent with the physics that determine fuel use. This approach allows us to leverage physics-based formulas while also flexibly allowing for deviations from them due to factors that are unaccounted for or inaccurately measured.

Using our hybrid approach, we are able to achieve a high degree of accuracy in predicting fuel consumption. For our data set comprising the global fleet of dry bulk ships, our best-performing model achieves an out-of-sample \ac{R2} of 0.954 and \ac{MAPE} of 10.9\%.\footnote{Our training set consists of data from 2019 and 2020, and our hold-out test sample is 2021 data.} This is a significant improvement over a purely calculation-based approach, which achieves an \ac{R2} of 0.58 and an \ac{MAPE} of 20.4\%. We evaluate both linear models (\ac{OLS}, lasso, and ridge) and tree-based models (random forest, gradient boosting Regression, and CatBoost). While ridge regression performs the best in terms of \ac{R2} for our hold-out sample, we find that all models perform quite well, with even the worst-performing achieving an \ac{R2} of 0.931 (\ac{MAPE}=12.3\%).
% Value sources: ML_FC_Fm2_test_fc.csv, generated by ML_FC.ipynb

% Contribution and results preview:
% - Machine learning provides a flexible functional form that can:
%     - Capture divergence from engineering relationships
%     - Help to mitigate error from imperfect/incomplete tracking data
%     - predict based on speed (OECD does not!)
% - Bottom-up vs. top-down
%     - How do reported values compare?
%     - How do our estimates compare?

The primary contribution of this work is a highly accurate methodology for predicting fuel consumption that can be used to produce timely estimates of shipping emissions under varying ship activity, such as travel speed. To our knowledge, this is the first study to directly predict fuel consumption using a hybrid of the industry-standard bottom-up calculation method and machine learning and apply it at a large geographic scale. The closest work to ours is that of \textcite{yan2023analysis} and \textcite{clarke2023co2} who both apply machine learning models to predict a ship-level annual average \ac{CO2} emission ratio with units of mass-per-distance-travelled. They obtain total emissions by multiplying this efficiency metric by the observed or reported distance travelled. Employing this approach for prediction relies on the assumption that fuel efficiency ``averages out'' over various operating conditions (e.g., speed) during a year for which fuel consumption is reported. While this may provide a satisfactory approximation when ship speeds remain similar to those used to train the model, it is in general biased due to the nonlinearity of fuel consumption with speed. As such, the error may be large when attempting to predict emissions under scenarios in which ships adjust their speed of travel, for example in response to emissions regulations.

We propose an alternative prediction methodology that has the similar data requirements for prediction, but leverages engineering and physics relationships to improve prediction performance across a wider range of operating conditions. We construct a predictor variable representing annual ship-level energy use by summing theoretical hourly energy use, which is calculated with the industry-standard Admiralty formula \parencite[p.~64]{faber2020fourth}. This formula relies on ship characteristics and hourly speed and draft observations. To allow for potential inaccuracies in the functional form, we also construct slight variations on this quantity and provide these as additional predictor variables.

% Clarke 2023: "The emissions efficiency ratio variable average CO2 emissions per nautical mile from the EU-MRV dataset was chosen as the target variable for the random forest model due to having high coverage in the EU-MRV dataset, and the ability to adapt it according to distances travelled, a variable that we can obtain using the AIS. In addition, ratio indicators provide more stability than level indicators, for example total CO 2 emissions. In this case, efficiency as defined by emissions per distance travelled also benefits from ease of interpretation. The assumptions underpinning this approach are that the technical specifications of a vessel stay constant in the short term, that differences in emissions efficiency during a voyage average out, and that changes in a ship’s emissions come principally from the amount of distance covered in a given time period. In this way, we exploit ship characteristics to determine emission patterns, while using real-time movement
% information of the AIS dataset to calculate timely and frequent estimates of ships’ emissions. The random forest model departs from equation (1) above by learning emission patterns such that a ship’s full real-time development, including its engines’ status and operational mode as provided by AIS, do not need to be
% explicitly modelled. This approach builds on a similar strategy used by team Blue Carbon from the Wärtsilä Corporation25 in the 2020 UN Hackathon for AIS data."


%  Why does this work better?
%  reduces demand on machine learning model to capture physics of fuel consumption
%  potential inaccuracies in formulas
%  potential data errors that we can condition on
Our methodology balances the advantages of both calculation-based and pure machine learning approaches. Compared to other machine learning approaches, our use of engineering- and physics-based relations to engineer predictor features means we rely less on the machine learning algorithms to capture the physics of ship fuel consumption. The fact that linear models (in logs) perform similarly to non-linear, tree-based models in our analysis is consistent with this conjecture. In contrast to a purely calculation-based approach, incorporating machine learning allows the model to capture deviations from theoretical relationships, as well as incorporate further information that may have predictive power, but for which there is no theoretical foundation to suggest a functional form. One example of this is our inclusion of additional features derived from \ac{AIS} tracking data that provide information on both ship behaviour and potential data errors, which is a well-known issue researchers face when using this data.

% Literature:
% - Bottom-up approach
%     - IMO Fourth GHG Study \parencite{faber2020fourth}
% - Bottom-up approach + MRV data
% - Shipping Emissions and Machine Learning
%     - OECD \cite{clarke2023co2}

This paper is related to three strands of literature. The first of these has developed various bottom-up emissions estimation methodologies using engineering calculations applied to \ac{AIS} tracking data (e.g., \cite{jalkanen2009modelling,olmer2017greenhouse,moreno2019comparative,tvete2020modelling}). We build directly from these techniques in constructing our predictor variables. In particular, we follow the methodology used in the most recent \ac{IMO} emissions report as closely as possible \parencite{faber2020fourth}. A second strand of literature has begun to utilize annual emissions reports from the \ac{EU}'s \ac{MRV} program to validate emissions estimates obtained with the bottom-up approach. The largest study in scope is the \ac{IMO}'s analysis, however only the first year of \ac{MRV} reporting was available at the time \parencite{faber2020fourth}. Subsequently, a handful of authors have provided similar analyses, albeit with very limited geographical scope (e.g., \cite{doundoulakis2022comparative,mannarini2020eu,hensel2020green,wu2023evaluation}). In contrast, we analyse emissions from all dry bulk ships entering the \ac{EU}, and provide a methodology for global estimates. Furthermore, rather than using the \ac{MRV} data to simply validate predictions, we employ it to improve upon previous estimation methodologies. Lastly, a recent strand of literature applies various machine learning techniques to improve shipping emissions estimates. The majority of these use high-frequency fuel consumption data, but focus on small geographic areas and/or numbers of ships (e.g. \cite{ren2022container,wang2023ship,hu2019prediction,jebsen2020estimating,monisha2023step}). \textcite{guo2022combined} use machine learning primarily to model the physics of weather effects on ship resistance in a computationally feasible manner, thereby extending the application of previous calculation-based model from \textcite{tvete2020modelling}. Our work employs readily available annual fuel consumption data, which allows us to study emissions on a much larger scale. As described above, our work is similar in objective to that of \textcite{yan2023analysis} who employ gradient boosting and \textcite{clarke2023co2}, who use random forest regression to predict ship-level fuel \textit{efficiencies}. We directly predict fuel consumption, which allows us to leverage predictors derived from engineering calculations, and thereby obtain very high prediction accuracy for fuel consumption and emissions.
