%-------------------------------------
% CONFIGURATION
%-------------------------------------

\documentclass[hidelinks,11pt]{article}

% TABLE PATHS
\makeatletter
\def\input@path{{../src/tables/}}
\makeatother

% JUSTIFICATION
\usepackage{geometry}
\geometry{
	paper=letterpaper, % Change to letterpaper for US letter
	top=2.5cm, % Top margin
	bottom=2.5cm, % Bottom margin
	left=2.5cm, % Left margin
	right=2.5cm, % Right margin
	%showframe, % Uncomment to show how the type block is set on the page
}
\usepackage{microtype} % Improve justification
\usepackage[onehalfspacing]{setspace}

% % SECTIONS, REFERENCES, AND CITATIONS
\usepackage[hyperfootnotes=false]{hyperref}
\hypersetup{
    colorlinks=false, %links not colored
    linktoc=all %link to both section and subsections
}
\usepackage[english]{babel}
\addto\extrasenglish{
	\def% -------------------------------------------------
\sectionautorefname{Section}
  \def\subsectionautorefname{Section}
	\def\subsubsectionautorefname{Section}
}
\usepackage[backend=biber,style=apa,autocite=inline,hyperref=false]{biblatex} 
\usepackage{csquotes}
\DeclareLanguageMapping{english}{english-apa}
\addbibresource{shippingemissionstrade.bib}
% Use shortauthor when available
\makeatletter
\AtEveryCitekey{\ifnameundef{shortauthor}{}{\def\cbx@apa@ifnamesaved{\@firstoftwo}}}
\makeatother

% MATH FORMATTING
\usepackage{amsmath}
\usepackage{mathtools}
\usepackage{amssymb}

% GRAPHICS
\usepackage{graphicx} % Allows including images
\graphicspath{{../src/plots/}}
\usepackage{float} % Allows for control of float positions

% TABLE FORMATTING
\usepackage[table]{xcolor} % for color in tables
\usepackage{booktabs}
\usepackage{multirow}
\usepackage{makecell}
\renewcommand\theadalign{cc}
\renewcommand\theadfont{\bfseries}
\renewcommand\theadgape{\Gape[1pt]}
\renewcommand\cellalign{cl}
\renewcommand\cellgape{\Gape[1pt]}
\usepackage{threeparttable} % regression notes on tables
\usepackage{adjustbox} % adjust table width

% % GENERAL
\usepackage{enumerate}

% TITLE PAGE
\newcommand{\acknowledgement}{Corresponding author. \\
    Vancouver School of Economics, University of British Columbia, 
    6000 Iona Drive, Vancouver, BC, Canada.\\
    \textit{Email addresses:} apeters@protonmail.com (A. Peters), hkasahar@mail.ubc.ca (H. Kasahara), xjy099@student.ubc.ca (O. Xu), zhaoxinbo1999@outlook.com (J. Zhao).\\
    We thank ... for excellent research assistance. Data acquisition was funded by a generous grant from the Centre for Innovative Data in Economics Research.
}
\title{Improving Maritime Shipping Emissions Estimates Using Machine Learning}
\author{Allen Peters\thanks{\acknowledgement}, Hiroyuki Kasahara, Oliver Xu, Jasper Zhao\\
        \vspace{1em}\\        
        {\Large \textbf{DRAFT}}\\
        % {\small \color{blue}\href{https://alpeters.github.io/assets/pdf/}{CLICK HERE FOR THE MOST RECENT VERSION}}
        }
\date{\today}

%---------------------------------------
% DOCUMENT
%---------------------------------------

\begin{document}

\begin{titlepage}

\maketitle

\begin{abstract}
    We explore the potential of machine learning algorithms to improve upon engineering estimates of CO$_2$ emissions from maritime shipping. Traditional estimates rely on engineering approximations that may not entirely capture actual fuel use. We match reported annual ship-level emissions from a European Union emissions reporting program with tracking data and technical characteristics for the global fleet of dry bulk ships. As a baseline, we follow industry standard procedures to calculate engineering estimates of annual ship-level emissions. We then train various machine learning algorithms on the residual---the discrepancy between reported and calculated emissions---and are able to improve out-of-sample prediction.
\end{abstract}


% % \vfill
% % \textit{Keywords:} 
% % \textit{JEL codes: }
% % \textit{JEL codes (2-digit): }

\thispagestyle{empty} % Removes page number from first page
\end{titlepage}

% % -------------------------------------------------

\section{Introduction}

Motivation:
    \begin{itemize}
        \item Shipping emissions are a significant contributor to global CO$_2$ emissions.
        \item Traditional estimates rely on engineering approximations that may not entirely capture actual fuel use.
        \item Higher frequency estimates (typically published with lag)
        \item Existing estimates may suffer from poor data quality
    \end{itemize}
What we do:
    \begin{itemize}
        \item Provide a machine learning approach to more accurately estimate CO$_2$ emissions
        \item Higher frequency estimates (typically published with lag)
        \item Take emissions reports as truthful and accurate
        \item Predict residual between reported and calculated emissions - thereby guarantee improvement upon standard bottom-up approach
        \item Dealing with missing data for extrapolation
        \item We are able to explain x\% of variation in residual
        \item Better untracked (no AIS data) predictions???
    \end{itemize}
Contribution and results preview:
    \begin{itemize}
        \item Machine learning provides a flexible functional form that can:
        \begin{itemize}
            \item Capture divergence from engineering relationships
            \item Help to mitigate error from imperfect/incomplete tracking data
        \end{itemize}
        \item Bottom-up vs. top-down
        \begin{itemize}
            \item How do reported values compare?
            \item How do our estimates compare?
        \end{itemize}
    \end{itemize}
Literature:
    \begin{itemize}
        \item IMO Fourth GHG Study
        \item precursors for estimation method: 
        \begin{itemize}
            \item 
        \end{itemize}
        \item Paper on detecting misreporting in EU MRV
        \item Other literatures???
    \end{itemize}


\section{Data}
\begin{itemize}
    \item AIS data
    \begin{itemize}
        \item Descriptive statistics on: missing observations, ...?
    \end{itemize}
    \item Vessel characteristics from Clarkson Research WFR
    \item EU MRV
    \begin{itemize}
        \item Context
        \item Potential for misreporting
        \item descriptive statistics on number of reporting ships, comparison of reporting to non-reporting over various dimensions (which ones?)
    \end{itemize}
\end{itemize}

\section{Methodology}
\begin{itemize}
    \item Matching AIS data to EU trips
    \begin{itemize}
        \item Compare to page 140 of IMO Fourth GHG Study \parencite{faber2020fourth}
    \end{itemize}
    \item Replicating IMO bottom-up approach (Highlight differences)
    \begin{itemize}
        \item Interpolating missing data
    \end{itemize}
    \item Machine Learning
    \begin{itemize}
        \item Variable selection
        \item Algorithms
        \begin{itemize}
            \item Linear: lasso, ridge, (elastic net?)
            \item discuss why not neural net
            \item 
        \end{itemize}
        \item Hyper parameter tuning (cross-validation)
    \end{itemize}
    \item Extrapolation
    \begin{itemize}
        \item Comparison of MRV reporting fleet to non-reporting
    \end{itemize}
\end{itemize}


\section{Results}
\begin{itemize}
    \item Linear vs. non-linear models
    \item discuss log-additive discrepancy
    \item Why do some models work better than others?
\end{itemize}

\section{Conclusion}



\end{document}